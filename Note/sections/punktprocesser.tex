I eksempel \ref{ex:car2} med trafikken i et vejkryds blev tiden mellem to bilers ankomst modelleret med en eksponentielfordelt tilfældig variabel $T$ med parametren $\lambda = 1/60^2$ s$^{-1}$. Lad nu $T_1, T_2, \dots$ være forskellige tilfældige variable der modellerer tiden mellem hver bil over en tidsperiode, eksempelvis en dag, således $T_i \sim \exp(\lambda)$ for $i = 0,1,\dots$. Ankomsttidspunkter over en hel dag er da en såkaldt \emph{punktprocess}, hvor hvert punkt markerer ankomstentidspunktet for en bil (se figur \ref{fig:process1}).
\begin{figure}[H]
\centering
\includegraphics[width = \textwidth]{process1.png}
\caption{Punktprocess hvor $\times$ markerer hændelser i tid, eksempelvis ankomsttidspunkter af biler i et trafikkryds.} \label{fig:process1}
\end{figure}
I det specielle tilfælde hvor $T_1, T_2, \dots$ er eksponentielfordelte med parameter $\lambda$, da er punktprocessen den såkaldte \emph{Poisson process}. Poisson processen bruges til at modellere processer hvor ankomster i tid sker tilfældigt og uafhængigt af hinanden. Andre eksempler kunne være ankomst af opgaver til en computerserver, kø på apoteket, tidspunkter for jordskælv, ulykker på en motorvej og henfald af radioaktive partikler.
\\ \\  
Følgende resultat om Possion processer er nyttigt og giver den sammenhæng vi tidligere så mellem Poission og eksponentielfordelingen. Lad $X(t)$ være en tilfældig variabel der tæller antallet af punkter fra $0$ til $t$ i en Poisson process med parameter $\lambda$. Det viser sig da at $X(t)$ følger en Possion fordeling med parameteren $t \lambda $. Den forventede værdi af $X(t)$ er dermed $E[X(t)] = t \lambda$. $\lambda$ kan således tolkes som antal forventede hændelser per tidsenhed og kaldes af denne grund \emph{raten}. 
\begin{example}
I en lufthavn ankommer der hvert døgn gennemsnitligt $400$ fly. Flyene ankommer tilfældigt i løbet af døgnet og er ikke koordineret flyene imellem. Ud fra disse oplysninger kan vi antage en Poisson process for ankomsttidspunkter. Hvis $X(t)$ er antallet af fly i løbet af t timer vil denne følge en Poisson fordeling og:
\begin{align*}
E[X(24)] = 24\lambda = 400 \Rightarrow \lambda = 400/24 \approx 16.7,
\end{align*}
således $X(t) \sim \text{Poi}(16.7\cdot t)$ og $T_i \sim \exp(16.7)$. Tiden $T_i$ mellem to flyvere er da eksponentielfordelt med parameter $16.7$. Den forventede tid er $E[T_i] = 1/16.7 = 0.06$ timer $= 3.6$ minutter med en standardafvigelse også på $3.6$ minutter. 
\end{example}
Et andet nyttigt resultat er brugbart til simulering af Poisson processer. Vi har givet en Poisson process med rate $\lambda$. Givet at der er $N$ antal punkter i tidsperioden $[0,t]$ vil fordelingen for tidspunkterne være uafhængigt uniformt fordelte tilfældige variable i intervallet $[0,t]$. Kender vi $N$ kan man altså blot simulere $N$ punkter uniformt fordelt i intervallet $[0,t]$ for at simulere en Poisson process. Algoritme \ref{alg:poisprocss} opsummerer simulering af Poisson processeer baseret på disse resultater. Se \cite[240-247]{olofsson2012} for mere om Poisson processer. 
\begin{algorithm}
\begin{algorithmic}
\STATE Simuler $N \sim \text{Poi}(t\lambda)$ 
\FOR{$i = 0$ til $N-1$}
\STATE Simuler $Y_i \sim \text{unif}(0,t)$ \COMMENT{$Y_i$ er tidspunktet for en hændelse}
\ENDFOR 
\end{algorithmic}
\caption{Simulering af Poisson process med rate $\lambda$ mellem $0$ og $t$} \label{alg:poisprocss}
\end{algorithm}