\emph{Sandsynlighed} er et begreb der (mis)bruges i mange sammenhænge indenfor videnskab, økonomi, politik med mere, men helt grundlæggende er det en matematisk konstruktion med en præcis definition. Matematiks set er sandsynlighed baseret på sandsynlighedsmålet $P$, en funktion der bestemmer sandsynligheden for udfald i \emph{tilfældige eksperimenter}. Hvis et tilfældigt eksperiment har $N$ mulige udfald $A_1, A_2,\dots, A_N$ da vil $P(A_i)$ være sandsynligheden for udfaldet $A_i$.
\begin{example} \label{ex:terning1}
Vi kaster en fair 6-siddet terning med mulige udfald 
$$A_1 = \text{``slå en 1'er''}, \quad  A_2 = \text{``slå en 2'er''}, \quad \dots \quad, A_6 = \text{``slå en 6'er''}$$
Sandsynligheden for alle udfald er lige stor og vi har eksempelvis at \\
$$P(A_6) = P(``\text{slå en 6'er''}) = 1/6 \approx 0.17\%.$$
\end{example}
Funktionen $P$ er defineret ud fra en række matematiske egenskaber. Vigtigst er at den altid antager værdier mellem $0$ og $1$ således $0 \leq P(A) \leq 1$ for ethvert udfald $A$ i et tilfældigt eksperiment. Se den fulde definition i \cite[sektion 1.3]{olofsson2012}. 
\\ \\
For at formalisere notationen i tilfældige forsøg indfører vi begrebet \emph{tilfældige variable}. En tilfældig variabel er en funktion der omsætter udfald i et tilfældigt forsøg til talværdier.
\begin{example}
En tilfældig variabel $X$ tæller antal gange en mønt lander på krone i løbet af $3$ kast. Vi har så $X(PPK) = 1$, $X(KKP) = 2$, $X(KKK) = 3$ osv.
\end{example}
\begin{example} \label{ex:terning2} Lad $X$ være en tilfældige variabel, der beskriver talværdien af terningens udfald fra eksempel \ref{ex:terning1}, Vi har således $X(\text{``slå en 1'er''}) = 1$, $X(\text{``slå en 2'er''}) = 2$, osv. \\
Vi kan da skrive eksempelvis $P(X = 6) = 1/6$.
\end{example}
Hvordan man rent faktisk udregner sandsynligheder er en historie for en anden gang. Her vil vi nøjes med at se på eksempler hvor vi allerede kender sandsynlighederne for alle udfald karakteriseret ved såkaldte \textit{fordelingsfunktioner}. 