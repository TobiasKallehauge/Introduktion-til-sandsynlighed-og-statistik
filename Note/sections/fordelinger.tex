Før vi kan indføre fordelingsfunktioner skal vi kategorisere mellem \emph{diskrete} og \emph{kontinuerte} tilfældige variable. Eksemplerne vi har set indtil videre er diskrete da der er et \emph{tælleligt} antal udfald. Det også muligt at have en diskret variabel med tælleligt uendelig mange udfald så længe man kan associere associere udfaldene med en tællelig mængde såsom de ikke negative heltal $\{0,1,2,\dots\}$. Antallet af terningslag før der slås en 6'er et eksempel på en tællelig uendelig mænge da der ikke er en øvre grænse for antal slag. Kontinuerte tilfældige variable derimod er \textit{utællelige} og er typisk associeret med de reelle tal $\mR$ eller et interval heri. Eksempelvis er højden af en tilfældigt udvalgt person eller tiden det tager for et atom at henfalde radioaktivt begge kontinuerte tilfældige variable. 
\subsection{Diskrete fordelinger}
Vi indfører nu \emph{sandsynlighedsmassefunktionen}, på engelsk \textit{probability mass funktion} (pmf). En pmf $p_X$ er defineret som $p_X(x) = P(X = x)$ for en tilfældig variable $X$. Bemærk her at $X$ er den tilfældige variabel mens $x$ er et reelt tal.
\begin{example} \label{ex:terning3} I $X$ fra eksempel \ref{ex:terning2} med terningkast er $p_X(x) = 1/6$ for $x = 1,2,\dots, 6$ og vi har eksempelvis $P(\text{``slå en 6'er''}) = P(X = 6) = p_X(6) = 1/6$. 
\end{example}
\begin{example} \label{ex:terning4} For en tilfældig variabel $X$ har vi givet pmf'en:
\begin{align*}
p_X(x) = \begin{cases}
\frac{3}{6} & x = -1 \\
\frac{1}{6} & x = 0 \\ 
\frac{2}{6} & x = 1
\end{cases},
\end{align*}
og vi har eksempelvis at $P(X = 0) = p_X(0) = \frac{1}{6}$. Selvom vi ikke ved noget om hvilket tilfældigt forsøg $X$ stammer fra, ved vi ud fra $p_X$ alt om hvordan $X$ opfører sig fra et statistisk synspunkt. Vi siger derfor at $p_X$ karakteriserer $X$ fuldstændigt samt at $X$ følger fordelingen for $p_X$. 
\end{example}
Med en pmf kan man nemt lave beregninger, der omhandler delmængder af udfaldsrummet. 
\begin{example}
Givet pmf'en for $X$ i eksempel \ref{ex:terning3} med terningkast har vi eksempelvis:
\begin{align*}
P(\text{``Slå mindst 3''}) &= P(X \leq 3) = p_X(1) + p_X(2) + p_X(3) = \frac{1}{6} + \frac{1}{6} + \frac{1}{6} = \frac{1}{2} \\
P(\text{``Slå mere end 4''}) &= P(X > 5) = p_X(5) + p_X(6) = \frac{1}{6} + \frac{1}{6} = \frac{1}{3} \\ 
P(\text{``Slå mellem 1 og 6''}) &= P(X \in \{1,2,3,4,5,6\}) = \sum_{x = 1}^6 p_X(x) =  \sum_{x = 1}^6 \frac{1}{6} = 1
\end{align*}
I det sidste eksempel udregnes sandsynligheden for hele udfaldsrummet ``Slå mellem $1$ og $6$'' til $1$. Dette giver intuitivt mening, men er faktisk også en egenskab der definerer sandsynlighedsmålet. 
\end{example}
%
\subsection{Kontinuerte fordelinger}
Ved kontinuerte fordelinger snakker man istedet for pmf'er om \textit{sandsynlighedstæthedsfunktioner}, på engelsk \textit{probability density function} (pdf). For at forstå forskellen ser vi på følgende eksempel.
\begin{example} \label{ex:unif1}
En klassisk kontinuert fordelt tilfældig variabel er $X$ fordelt mellem $0$ og $1$ med lige stor sandsynlighed for alle værdier. Med hvad er så sandsynligheden for en bestemt værdi i intervallet, eks. $P(X = 0.5)$? Svaret er faktisk $0$, og det er fordi der er utælleligt mange mulige udfald og sandsynligheden for et specifikt udfald er altid $0$. Spørger man om sandsynligheder for intervaller i stedet kan man dog få ikke-nul sandsynligheder. Intuitivt giver det eksempelvis mening at $P(0.25 \leq X \leq 0.5) = 0.25$, men for at komme frem til dette skal det bruges at $X$ har pdf funktionen $p_X(x) = 1$ og vi får sandsynligheden ud fra følgende integrale:
\begin{align*}
P(0.25 \leq X \leq 0.5) = \int_{0.25}^{0.5} p_X(x) \ dx = \int_{0.25}^{0.5} 1 \ dx = \left[x \right]_{x = 0.5}^{0.25} = 0.5 - 0.25 = 0.25
\end{align*}
\end{example} 
For en kontinuert tilfældig variabel $X$ med pdf $p_X$ gælder der generelt:
\begin{align*}
P(a \leq X \leq b) = \int_a^b p_X(x) \ dx
\end{align*}
Med denne teori har vi et grundlag for sandsynlighed og vi vil nu bevæge os over i statistik. 