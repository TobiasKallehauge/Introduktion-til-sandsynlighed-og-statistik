
 En uniform fordeling kan også være kontinuert hvis vi tillader alle reelle tal i et interval $[a,b]$ og vi har pdf funktionen $p_X(x) = 1/(b-a)$\footnote{I kontinuerte fordelinger skal man integrere for at udregne sandsynligheder. Vi har specielt at \\
 $P(a \leq X \leq b) = \int_a^b p_X(x)\ dx$. Eksempelvis har vi for en kontinuert uniform fordelingen mellem $0$ og $1$ at\\ $X \sim \text{unif}(0,1)$, $p_X(x) = 1$ og $P(0 \leq X \leq 0.5) = \int_0^{0.5} 1 \ dx = 0.5$.}.
\\ \\ 



Det viser sig at mange forsøg kan kategoriseres med parametriske \textit{fordelinger} med tilhørende pmf'er. Fra kursusgangen om tilfældige tal kender vi allerede et par fordelinger nemlig den \textit{uniforme fordeling}  og \textit{normalfordelingen}. Forsøget med terningslag er et eksempel på en uniform fordeling hvor $p_X(x) = 1/6$ for $x = 1,\dots, 6$, altså lige stor sandsynlighed for alle udfald. I terningslag forsøget følger $X$ altså en uniform fordeling med udfald mellem $1$ og $6$ og vi skriver $X \sim \text{unif}(1,6)$. Forsøget med terning og møntkast er eksempler på \textit{diskrete fordelinger}, da der er et tælleligt antal udfald i forsøget.