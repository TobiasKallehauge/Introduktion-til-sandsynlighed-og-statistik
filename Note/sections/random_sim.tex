Der findes specielle algoritmer til at generere tilfældige tal på en computer. Dette kan bruges i eksempelvis en diskrettidssimulering hvor tilfældige tal \emph{trækkes} fra fordelinger baseret på statistiske modeller. Da de tilfældige tal genereres ud fra en deterministisk algoritme kalder vi dem for \emph{pseudo} tilfældige, men de kan dog opføre sig statistisk ligesom ægte tilfældige tal og er derfor brugbare. Lad os se på hvordan man kan generere et tilfældigt heltal i C med \texttt{rand} funktionen fra \texttt{stdlib}: 
\begin{lstlisting} 
#include <stdio.h>
#include <stdlib.h>

void main() {
    
    int N = 5; 
    unsigned int nr; 
    
    printf("Random numbers between 0 and %u: ", RAND_MAX); 
    
    for(int i = 0; i < N; i++) {
       nr = rand(); 
       printf("%u, ", nr);  
    }
    printf("\n"); 
}
\end{lstlisting}
\lstset{style=console,breaklines=true}
I konsollen får vi: 
\begin{lstlisting} 
$ rand_tutorial1.exe
Random numbers between 0 and 32767: 41, 18467, 6334, 26500, 19169,
\end{lstlisting}
Gentages eksemplet ser vi dog noget specielt: 
\begin{lstlisting} 
$ rand_tutorial1.exe
Random numbers between 0 and 32767: 41, 18467, 6334, 26500, 19169,
\end{lstlisting}
Vi får de samme tilfældige tal begge gange og det gør vi da tallene er genereret ud fra en deterministisk algoritme. Mere præcist baseres første kald af \texttt{rand} på et såkaldt \emph{seed}, et ikke negativt heltal der ``sætter gang'' i genereringen af tilfældige tal. For \texttt{stdlib} er seed som udgangspunktet $0$, og ændres dette ikke vil de samme tal altid genereres. Man kan manuelt sætte seed med \texttt{srand} funktionen fra \texttt{stdlib}. Ønsker man automatisk valg af seed kan man eksempelvis bruge \texttt{time} funktionen fra \texttt{time} biblioteket: 
\lstset{style = mystyle}
\begin{lstlisting} 
#include <stdio.h>
#include <stdlib.h>
#include <time.h>

void main() {
    
    int N = 5; 
    unsigned int nr;
    srand(time(0)); // Set seed to system time
     
    printf("Random numbers between 0 and %u: ", RAND_MAX); 
    
    for(int i = 0; i < N; i++) {
       nr = rand(); 
       printf("%u, ", nr);  
    }
    printf("\n"); 
}
\end{lstlisting}
\lstset{style = console}
\begin{lstlisting}
$ rand_totorial2.exe
Random numbers between 0 and 32767: 29042, 25101, 25927, 18539, 21689,
\end{lstlisting}
I videnskabeligt arbejde er det vigtigt at kontrollere seed manuelt, således resultater principielt kan genskabes. Det er også tit praktisk at kontrollere seed manuelt ved debugging af programmer. Læs mere om algoritmer til genering af tilfældige tal og seed i \cite[282-283]{olofsson2012}. 