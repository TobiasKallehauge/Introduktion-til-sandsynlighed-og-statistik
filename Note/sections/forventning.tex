I statistik forsøger vi at karakterisere fordelinger ud fra \emph{statistikker}, der kan fortælle os om vigtige egenskaber for disse. Den \emph{forventede værdi} er en sådan statistik, og som navnet hentyder fortæller den om det forventede udfald af et tilfældigt forsøg. For diskret tilfældig variabel  $X$ med mulige udfald $\{x_1,x_2,\dots\}$ og pmf $p_X$ er den forventede værdi defineret:
\begin{align*}
E[X] = \sum_{k=1}^{\infty} x_k\cdot p_X(x_k),
\end{align*}
altså en vægtet sum over alle mulige udfald.
\begin{example} \label{ex:game}
En skummel person på gaden tilbyder dig at gamble i et terningspil hvor du mister $1$ krone hvis du slår mindre end $4$, du vinder ingenting hvis du slår $4$ og du vinder $1$ krone hvis du slår $5$ eller mere. Burde du spille med? \\
For at vurdere dette starter vi med at definere den tilfældige variabel $X$, der er $-1$ hvis du slår mindre end $4$, $0$ hvis du slår $4$ og $1$ hvis du slår $5$ eller mere. Sandsynligheden for de tre udfald er henholdsvis $3/6$, $1/6$ og $2/6$ og derfor er pmf funktionen den samme som i eksempel \ref{ex:terning4}. Vi kan nu udregne det forventede udfald af spillet:
\begin{align*}
E[X] = -1\cdot p_X(-1) + 0\cdot p_X(0) + 1\cdot p_X(1) = -1\cdot \frac{3}{6} + 0\cdot\frac{1}{6} + 1\cdot \frac{2}{6} = -\frac{1}{6}
\end{align*}
Du forventes altså at miste $1/6$ krone hver gang du spiller og du anbefales herfra ikke at spille med. Bemærk som her, at det forventede udfald ikke nødvendigvis er blandt de mulige udfald. 
\end{example}
For kontinuerte tilfældige variabel er den forventede værdi defineret ud fra et integrale, men fortolkningen er den samme. Hvis $X$ er en kontinuert tilfældig variabel med mulige udfald i alle reelle tal og pdf $p_X$, da er den forventede værdi:
\begin{align*}
E[X] = \int_{-\infty}^{\infty} x \cdot p_X(x) \ dx
\end{align*}
\begin{example}
Den forventede værdi af $X$ fra eksempel \ref{ex:unif1} med pdf $p_X(x) = 1$ for udfald mellem $0$ og $1$ er:
\begin{align*}
E[X] = \int_0^1 x\cdot 1 \ dx = \left[\frac{1}{2}x^2 \right]_{x=0}^1 = \frac{1}{2}1^2 - \frac{1}{2}0^2 = \frac{1}{2}
\end{align*}
\end{example}
Det græske symbol $\mu$ bruges som oftest til at notere den forventede værdi altså $\mu = E[X]$. To andre meget brugbare statistikker er \emph{varians} og \emph{standardafvigelse}, der fortæller noget om hvor meget $X$ afviger fra sin forventede værdi. Varians for en tilfældig variabel med forventet værdi $\mu$ er defineret som kvadratet af den forventede afvigelse fra $\mu$: 
\begin{align*}
\Var[X] = E[(X - \mu)^2] = \begin{cases}
\displaystyle \sum_{k = 1}^{\infty} (x_k - \mu)^2p_X(x_k) & \text{for diskret } X \\[15pt]
 \displaystyle \int_{-\infty}^{\infty} (x - \mu)^2p_X(x) \ dx & \text{for kontinuert} X 
\end{cases}
\end{align*}
Varians er en positiv størrelse og er matematisk nem at arbejde med, men kan være lidt svær at fortolke. Hvis $X$ eksempelvis er en vægt i gram (g) med forventet værdi $\mu = 4$ g, da er en varians på  $\Var[X] = 4$ g$^2$ svær at fortolke. Derfor bruges standardafvigelsen, der fortæller om den forventede afvigelse fra middelværdien og er defineret ud fra varians:
\begin{align*}
\Std[X] = \sqrt{\Var[X]}
\end{align*}
\begin{example}
Variansen for spillet i eksempel \ref{ex:game} med $\mu = 1/6$ er:
\begin{align*}
\Var[X] &= (-1 - \mu)^2p_X(-1) + (0 - \mu)^2p_X(0) + (1 - \mu)^2p_X(1) \\
 &= (-1 + 1/6)^2\frac{3}{6} + (1/6)^2\frac{1}{6} + (1 + 1/6)^2\frac{2}{6} = \frac{174}{216} \approx 0.81
\end{align*}
Og standard afvigelsen er $\Std[X] \approx 0.90$. 
\end{example}
\begin{example}
Variansen for $X$ i eksempel \ref{ex:terning4} med forventet værdi $\mu = 1/2$ er:
\begin{align*}
\Var[X] = \int_0^1 (x - 1/2)^2 \cdot 1 \ dx = \left[\frac{1}{3}x^3 - \frac{1}{2}x^2 + \frac{1}{4}x \right]_{x = 0}^{1} = \frac{1}{12} \approx 0.08,
\end{align*}
med standard afvigelse $\Std[X] = 1/\sqrt{12} \approx 0.29$. 
\end{example}
Varians og standardafvigelse noteres typisk henholdsvis $\sigma^2$ og $\sigma$. Med forventet værdi, varians og standard afvigelse har vi de vigtigste statistikker til at forstå en lang række fordelinger. I næste afsnit skal vi se på en række specielle fordelinger. 