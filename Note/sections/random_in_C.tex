C er ikke ligefrem statistikerens første valg når der skal arbejdes med simuleringer og statistiske modeller, og det er derfor begrænset hvor mange biblioteker der findes til at trække tal fra forskellige fordelinger især de mere avancerede som Poisson og eksponentielfordelingen\footnote{I min søgen har jeg ikke kunne finde andre biblioteker end stdlib, så skriv endelig til mig hvis du kan finde et godt bibliotek.}. Heldigvis kan vi basere genering af tilfældige ud fra \texttt{rand} funktionen i kombination med passende algoritmer for at simulere fra den ønskede fordeling. Der gives her algoritmerne til at trække fra fordelingerne beskrevet i denne note, og det er så op til jer at implementere i C. 
\subsection{Tilfældigt tal mellem $0$ og $N$}
Vi har allerede set hvordan man genererer tilfældige tal mellem \texttt{0} og \texttt{RAND\_MAX} som typisk er \\ 
$32764 = 2^{15} - 1$. Algoritme \ref{alg:zeroN} beskriver hvordan et tilfældigt tal mellem $0$ og $N \leq$ \texttt{RAND\_MAX} kan genereres.
\begin{algorithm}[H]
\begin{algorithmic}
\STATE Simuler $Y$ mellem $0$ og \texttt{RAND\_MAX} med \texttt{rand}. 
\STATE $X \gets Y \mod (N + 1)$ \COMMENT{$\mod$ er modulus operatoren som er \texttt{\%} i C}
\end{algorithmic}
\caption{Tilfældigt tal $X$ mellem $0$ og $N \leq$ \texttt{RAND\_MAX}} \label{alg:zeroN}
\end{algorithm}
\subsection{Uniform fordeling}
Simulering af kontinuert uniformt fordelte tal mellem $0$ og $1$ danner base for simulering af alle andre fordelinger, algoritme \ref{alg:unif01} beskriver hvordan. 
\begin{algorithm}[H]
\begin{algorithmic}
\STATE Simuler $Y$ mellem $0$ og $N$
\STATE $X \gets Y/N$
\end{algorithmic}
\caption{Uniform tilfældige variabel $X$ mellem $0$ og $1$} \label{alg:unif01}
\end{algorithm}
Teknisk set er $X$ i algoritme \ref{alg:unif01} diskret uniformt fordelt i mængden $\left\{0, \frac{1}{N},\dots,\frac{N-1}{N},N \right\}$, men for tilpas høj $N$ er dette en acceptabel tilnærmelse. Algoritme \ref{alg:unif02} beskriver simulering af en uniformt tilfældig variabel i et vilkårligt interval $[a,b]$. 
\begin{algorithm}[H]
\begin{algorithmic}
\STATE Simuler $U \sim \text{unif}(0,1)$
\STATE $X \gets U(b-a) + a$
\end{algorithmic}
\caption{Uniform tilfældige variabel $X$ mellem $a$ og $b$} \label{alg:unif02}
\end{algorithm}
\subsection{Diskrete fordelinger}
Givet en pmf kan alle diskrete fordelinger, herunder Bernoulli og Poisson, simuleres med samme algoritme. Bernoulli fordelingen er dog speciel simpel at simulere og fortjener sin egen algoritme - algoritme \ref{alg:bernoulli}. 
\begin{algorithm}[H]
\begin{algorithmic}
\STATE Simuler $U \sim \text{unif}(0,1)$
\IF{$U > p$}
\STATE $X \gets 1$
\ELSE
\STATE $X \gets 0$
\ENDIF
\end{algorithmic}
\caption{Bernoulli fordeling med parameter $p$} \label{alg:bernoulli}
\end{algorithm}
Den generelle algoritme til simulering af tilfældige tal fra diskete fordelinger med pmf $p_X$ og mulige udfald $x_1, x_2,\dots$ er givet i algoritme X
\begin{algorithm}[H]
\begin{algorithmic}
\STATE Simuler $U \sim \text{unif}(0,1)$
\IF{$U > p$}
\STATE $X \gets 1$
\ELSE
\STATE $X \gets 0$
\ENDIF
\end{algorithmic}
\caption{Bernoulli fordeling med parameter $p$} \label{alg:bernoulli}
\end{algorithm}
\subsection{Eksponentiel fordelingen}
\subsection{Normalfordelingen}